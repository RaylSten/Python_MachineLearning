\addcontentsline{toc}{chapter}{序}
\chapter*{序}

\LaTeX 是在 1980 年代開發出來的一種文件排版系統,可以用來編輯要出版的研究論文、書籍等文件,甚至也可以拿來製作簡報用的投影片,而 \LaTeX 跟其他如 Microsoft Word 之類的軟體最大的不同點在於 Microsoft Word 直接編輯檔案並且直接呈現編輯後的結果,而 \LaTeX 用起來有點像在編寫程式。用 \LaTeX 編輯的文件通常都會比用 Microsoft Word 編輯的文件美觀一點,尤其是在像資訊科學、數學、工程學等需要呈現數學公式的文件中更是如此。這項差異主要是因為 \LaTeX 用的字體不同,而且用在數學符號上特別好看,最後我覺得 \LaTeX 最強大的點在於編纂參考書目,其中最常見的是使用 \verb|.bib| 檔,這種格式的檔案可以提供跟文檔相容的參考書目列表。也因為種種上述原因讓我在撰寫這本書上更加有動力。\\

本書共分 7 個章節,大抵分成 : 「\LaTeX 入門實用指南」、「Python 基礎繪圖與繪圖技巧」、「Python 機率分配與隨機亂數」、「迴歸學習器之探討」、「判別式分析與 K-近鄰演算法之探討」、「類神經網路學習器之探討」、「學習器的評比」,前三個章節主要把重點放在 \LaTeX 的編排撰寫與 Python 的語法上,而後四個章節進入淺度機器學習的領域,第四到第六章節主要介紹各種學習器之理論、分群方法與實作,最後透過生成模擬資料的訓練評比學習器,最後一個章節透過生成兩個群組及三個群組的模擬資料訓練評比線性判別式分析、二次判別式分析、K-近鄰演算法以及類神經網路等六種學習器。

\begin{flushright}
    陳柏維
    \par\vspace*{-2pt}\hfill 2022年1月於臺北大學
\end{flushright}
