%\input{../preamble_math}   
%\title{\shadowbox{Python 基礎繪圖與繪圖技巧 \,}}
%\author{\small 陳柏維\footnote{E-mail: {\it s711033120@gm.ntpu.edu.tw}} \\
%{\small \it 國立臺北大學統計研究所}}
%\date{\small \today}
%\begin{document}
%\maketitle
%\fontsize{12}{20pt}\selectfont
%\setlength{\parindent}{0pt}

\chapter{Python 基礎繪圖與繪圖技巧}
本文將介紹 Python 的繪圖套件「Matplotlib」,透過更改 Matplotlib 的預設參數,例如 : 顏色、樣式、粗細等,搭配散佈圖及折線圖,達到更好的呈現效果。

\section{Matplotlib 介紹}
Matplotlib 是一個 Python 的繪圖套件,它可以將資料視覺化,使我們未來在做資料分析時比較好觀察資料的走向。
\subsection{圖例}

圖例用於說明繪製在圖形上的資料,透過元素或顏色來解釋資料。在 Matplotlib 圖形中,可以使用 \verb|plt.legend()| 函式新增圖例。
\bigskip
\begin{lstlisting}[language = Python]
import numpy as np
import matplotlib.pyplot as plt

x = np.linspace(-3, 3, 100)
f1 = lambda x : np.exp(x)
f2 = lambda x : np.exp(-x)
fig, ax = plt.subplots(figsize=(5,4))
ax.plot(x, f1(x), c = 'r', label = r'$y=e^x$')
ax.plot(x, f2(x), c = 'g', label = r'$y=e^{-x}$')
plt.legend()
plt.show()
\end{lstlisting}
\begin{figure}[H]
\centering
\includegraphics[scale=0.6]{\imgdir legend_1.eps}
\includegraphics[scale=0.6]{\imgdir legend_2.eps}
\caption{Exponential function}
\label{fig:legend}
\end{figure}

\verb|plot| 函式中的 \verb|linewidth| 引數可用於控制特定物件的繪圖寬度,\verb|set_linewidth()| 方法可用於控制 Matplotlib 中圖例的線寬。如圖 \ref{fig:legend} 所示。
\bigskip
\begin{lstlisting}[language = Python]
ax.plot(x, f1(x), c = 'r', label = r'$y=e^x$', \
	linewidth = 2.0) 
ax.plot(x, f2(x), c = 'g', label = r'$y=e^{-x}$', \
	linewidth = 4.0)
leg = plt.legend()
leg.get_lines()[0].set_linewidth(6) # 調整圖例線寬
leg.get_lines()[1].set_linewidth(10)
\end{lstlisting}

圖例可以透過使用 \verb|bbox_to_anchor| 放置在 Matplotlib 中的繪圖之外。\verb|bbox| 表示容納圖例的邊界框 - \verb|bounding box|。若未呼叫 \verb|tight_layout()|,則圖例框將被裁剪。而利用此參數時,需重新調整圖片之寬度與長度,如圖 \ref{fig:legend_1} 所示。
\bigskip
\begin{lstlisting}[language = Python]
fig, ax = plt.subplots(figsize=(6,4))
ax.plot(x, f1(x), c = 'r', label = r'$y=e^x$')
ax.plot(x, f2(x), c = 'g', label = r'$y=e^{-x}$')
plt.legend(bbox_to_anchor=(1.05, 1.0), loc='upper left')
plt.tight_layout()
\end{lstlisting}

\begin{figure}[H]
    \centering
        \includegraphics[scale=0.6]{\imgdir legend_3.eps}
    \caption{Exponential function 1}
    \label{fig:legend_1}
\end{figure}
\subsection{散佈圖}
在 Matplotlib 中,我們可以使用 \verb|plt.scatter()| 函式繪製散佈圖。透過函式 \verb|scatter()| 中的 \verb|s| 控制散點圖中標記的大小,如圖 \ref{fig:size} 所示。
\begin{lstlisting}[language = Python]
x = [1,2,3,4,5]
y = [a**2 for a in x]
s = [10*4**n for n in range(len(x))]
plt.scatter(x, y, s = s, color = 'purple')
plt.title('Doubling width of marker in scatter plot')
plt.xlabel('x')
plt.ylabel(r'$x^2$')
\end{lstlisting}
\begin{figure}[H]
    \centering
        \includegraphics[scale=0.6]{\imgdir size.eps}
    \caption{Scatter plot marker size}
    \label{fig:size}
\end{figure}
透過函式 \verb|projection = 'polar'| 可將座標改為極座標,圖 \ref{fig:polar} 為利用隨機生成之半徑及弧度所畫出的圖形,用以比較直角座標與極座標之差異。

\begin{lstlisting}[language = Python]
N = 100
r = 2 * np.random.rand(N) 
theta = 2*np.pi*np.random.rand(N)
area = 8*r**2 				      
colors = theta                    
ax = plt.subplot(211)
c = ax.scatter(theta, r, c = colors, s = area, cmap = 'hsv')
ax = plt.subplot(212, projection = 'polar') # 轉換極座標
c = ax.scatter(theta, r, c = colors, s = area, cmap = 'hsv')
plt.subplots_adjust(wspace = 0.4, hspace = 0.4)
\end{lstlisting}
\begin{figure}[H]
    \centering
        \includegraphics[scale=0.95]{\imgdir polar.eps}
    \caption{Polar scatter plot}
    \label{fig:polar}
\end{figure}

透過函式 \verb|scatter()| 中的 \verb|marker| 更改散點圖中標記圖案,並透過 \verb|marker = '$ $'|定義出非常規之圖形,圖 \ref{fig:marker} 為撲克牌的四個花色。
\bigskip
\begin{lstlisting}[language = Python]
x = [1,2,3,4]
y1 = [i**2 for i in x]
y2 = [i**2+5 for i in x]
y3 = [7*i+3 for i in x]
y4 = [7*i+7 for i in x]
plt.scatter(x, y1, marker = '$\diamondsuit$', \
	color ='r', s = 100, label = r"$x^2$")
plt.scatter(x, y2, marker = '$\heartsuit$', \
	color ='r', s = 100, label = r"$x^2+5$")
plt.scatter(x, y3, marker = '$\clubsuit$', \
	color = 'black', s = 100, label = r"$7x+3$")
plt.scatter(x, y4, marker = '$\spadesuit$', \
	color = 'black', s = 100, label = r"$7x+7$")
\end{lstlisting}
\begin{figure}[h]
    \centering
        \includegraphics[scale=0.6]{\imgdir marker.eps}
    \caption{Scatter plot marker}
    \label{fig:marker}
\end{figure}

透過函式 \verb|plt.clim()| 在 Matplotlib 中設定顏色條的範圍。如圖 \ref{fig:clim} 所示。
\bigskip
\begin{lstlisting}[language = Python]
import random
s_x = random.sample(range(0, 100), 20)
s_y = random.sample(range(0, 100), 20)
s = plt.scatter(s_x,s_y,c = s_x, cmap = 'plasma')
c = plt.colorbar()
plt.clim(0, 150) 
\end{lstlisting}
\begin{figure}[h]
    \centering
        \includegraphics[scale=0.6]{\imgdir clim.eps}
    \caption{Scatter plot cilm}
    \label{fig:clim}
\end{figure}

在 Matplotlib 中若想將散佈圖上的點用線連接,可以透過 \verb|zorder| 來設定 Matplotlib 圖中的繪製順序來設定 Matplotlib 圖中的繪製順序,將 \verb|plot| 和 \verb|scatter| 分配不同的順序,然後顛倒順序以顯示不同的繪製順序行為,如圖 \ref{fig:point_line} 所示。
\bigskip
\begin{lstlisting}[language = Python]
x = np.linspace(0, 5, 50)
y = np.exp(2*x)
plt.scatter(x, y, color = 'r', s = 25, zorder = 1)
plt.plot(x, y, color = 'b', zorder = 2)
\end{lstlisting}
\begin{figure}[H]
    \centering
        \includegraphics[scale=0.6]{\imgdir point_line.eps}
    \caption{Connected Scatterplot points with line}
    \label{fig:point_line}
\end{figure}

\subsection{折線圖}
在 Matplotlib 中,可以使用 \verb|plt.plot(x, y, 'style_code')| 函式繪製折線圖。
\begin{itemize}
\item
x : position along the x-axis
\item
y : position along the y-axis
\item
style code : abbreviation code for line color, line style, and marker style
\end{itemize}
圖 \ref{fig:multiple_lines} 為折線圖各種面貌之展現,如線的顏色、樣式、粗細。
\bigskip
\begin{lstlisting}[language = Python]
x = np.linspace(-2, 2, 50)
y1 = np.exp(x)
y2 = np.exp(-x)
y3 = x**2+5
y4 = -x**2-2
plt.plot(x, y1, 'k--o', \
	linewidth = 1.0) # black dashed line with circle marker
plt.plot(x, y2, 'g:d', \
	linewidth = 3.0) # green dotted line with dimond marker
plt.plot(x, y3, 'r*', \
	linewidth = 5.0) # red "*" marker (no line)
plt.plot(x, y4, 'm-.', \
	linewidth = 7.0) # magenta dot-dashed line (no marker)
\end{lstlisting}
\begin{figure}[H]
    \centering
        \includegraphics[scale=0.65]{\imgdir multiple_lines.eps}
    \caption{Multiple lines in one figure}
    \label{fig:multiple_lines}
\end{figure}

利用以下指令,可以在圖的不同位置加上文字 : 
\begin{itemize}
\item
\verb|plt.title('string', fontsize = n)|: 加標題
\item
\verb|plt.xlabel('string', fontsize = n)|: 加 x 軸名稱
\item
\verb|plt.ylabel('string', fontsize = n)|: 加 y 軸名稱
\item
\verb|plt.text(xp, yp, 'string', fontsize = n)|: 在 (\verb|xp|, \verb|yp|) 的位置加入文字
\end{itemize}
自訂座標軸刻度與刻度上的標記文字
\begin{itemize}
\item
\verb|plt.xticks([array of tick],[list of tick labels]) |: x 軸刻度、標記
\item
\verb|plt.yticks([array of tick],[list of tick labels]) |: y 軸刻度、標記
\item
\verb|plt.xticks| 要寫在 \verb|plt.xlim| 之前,否則會被 \verb|xlim| 的設定蓋過
\end{itemize}
圖 \ref{fig:axis} 將以上函數結合至一張圖上。
\bigskip
\begin{lstlisting}[language = Python]
x = np.linspace(0, 5, 30)
y = x**2 + 2
plt.plot(x, y, 'y--')
plt.xticks(np.linspace(0, 5, 11)) 
plt.xlim([0, 5])
plt.ylim([0, 30])
plt.xlabel('X')
plt.ylabel('Y')
plt.title(r'$y = x^2 + 2$')
plt.text(2, 20, r'$y = x^2 + 2$', fontsize = 18)
\end{lstlisting}
\begin{figure}[H]
    \centering
        \includegraphics[scale=0.5]{\imgdir axis.eps}
    \caption{Change the axis and tick mark}
    \label{fig:axis}
\end{figure}

Python 預設使用線性座標 (linear),要調整為對數座標 (log) 需利用 \verb|plt.xscale('log')| 函式,而若要指定以 k 為底則需在後面加入 \verb|base = k|,如圖 \ref{fig:log} 所示。
\bigskip
\begin{lstlisting}[language = Python]
x = [10, 100, 1000, 10000, 100000]
y = [2, 4 ,8, 16, 32]

plt.scatter(x, y, color = 'r')
plt.plot(x, y, 'c')
plt.grid()
plt.xscale('log')
plt.yscale('log', base = 2)
\end{lstlisting}
\begin{figure}[H]
    \centering
        \includegraphics[scale=0.65]{\imgdir log.eps}
    \caption{Plot with both log axes}
    \label{fig:log}
\end{figure}

透過 \verb|plt.subplots(m, n)| 新增子圖,將列數 \verb|m| 和行數 \verb|n| 作為引數傳遞返回一個圖物件和軸物件,用來操作圖形。使用 \verb|subplots()| 在 \verb|n| 行 \verb|m| 列的網格中生成一個有 \verb|m x n| 個子圖的圖形。第 1 個子圖利用 ax 列表中的第 1 個元素進行操作,第 2 個子圖利用 ax 列表中的第 2 個元素進行操作,依此類推。如果沒使用 \verb|tight_layout()| 函式,則該行座標軸將與下一行的標題重疊。圖 \ref{fig:subplot} 利用子圖的方式呈現四個著名的激勵函數。
\bigskip
\begin{lstlisting}[language = Python]
x = np.linspace(-3, 3, 100)
y1 = np.exp(x)
y2 = (np.exp(x) - np.exp(-x))/(np.exp(x) + np.exp(-x))
y3 = 1/(1 + np.exp(-x))
y4 = np.log(1 + np.exp(x))

fig, ax = plt.subplots(2, 2, figsize=(8, 6)) # 新增子圖
ax[0, 0].plot(x, y1, 'r')
ax[0, 1].plot(x, y2, 'c--')
ax[1, 0].plot(x, y3, 'g:')
ax[1, 1].plot(x, y4, 'y-.')

ax[0, 0].set_title("Exponential function")
ax[0, 1].set_title("Hyperbolic tangent function")
ax[1, 0].set_title("Sigmoid function")
ax[1, 1].set_title("Softplus function")
fig.tight_layout()
\end{lstlisting}
\begin{figure}[H]
    \centering
        \includegraphics[scale = 0.7]{\imgdir subplot.eps}
    \caption{Activation functions}
    \label{fig:subplot}
\end{figure}

\section{函數圖形練習}

\begin{enumerate}
\item
$f(x) = sin(x) + cos(x)$

注意事項 : 
\begin{itemize}
\item
繪製三角函數圖形必須留意其週期
\end{itemize}
\bigskip
\begin{lstlisting}[language = Python]
x = np.linspace(-3*np.pi, 3*np.pi, 200)
f1 = lambda x : np.sin(x)
f2 = lambda x : np.cos(x)
f3 = lambda x : np.sin(x) + np.cos(x)
fig, ax = plt.subplots()
ax.plot(x, f1(x), color = 'g', label = 'sin(x)')
ax.plot(x, f2(x), color = 'm', label = 'cos(x)')
ax.plot(x, f3(x), color = 'c', label = 'sin(x)+cos(x)')
ax.set_xticks(np.array([-3, -2, -1, 0, 1, 2, 3])*np.pi)
ax.set_xticklabels(['-3$\pi$', '-2$\pi$', '$\pi$', '0', \
  '$\pi$','2$\pi$','3$\pi$'], fontsize = 10, color = 'b')
ax.set_ylim([-3, 3])
ax.grid(True), ax.legend()
ax.set_xlabel('X'), ax.set_ylabel('f(x)')
ax.set_title('sin(x) and cos(x)')
\end{lstlisting}
\begin{figure}[H]
    \centering
        \includegraphics[scale=0.7]{\imgdir sincos.eps}
\end{figure}

\item
$\displaystyle f(x) = \frac{1-e^{-2x}}{1+e^{-2x}}$
\vspace{0.2cm}

注意事項 :
\begin{itemize}
\item
水平漸進線為 $y = 1$ 與 $y = -1$
\item
$(0, 0)$ 為反曲點(Inflection point)
\end{itemize}
\bigskip
\begin{lstlisting}[language = Python]
x = np.linspace(-10, 10, 100)
f = lambda x : (1 - np.exp(-2*x))/(1 + np.exp(-2*x))
fig = plt.figure()
plt.plot(x, f(x), color = 'c')
plt.hlines(y = 1, xmin = -12, xmax = 12, color = 'r', \
	linestyles = '--')
plt.hlines(y = -1, xmin = -12, xmax = 12, color = 'r', \
	linestyles = '--')
plt.text(0, 0, 'x', color = 'k', fontsize = 15, \
 	horizontalalignment = 'center', \
 	verticalalignment = 'center')
plt.grid(True)
plt.xlabel('X'), plt.ylabel('Y')
plt.ylim([-2, 2])
plt.title(r'$f(x)= \frac{1 - e^{-2x}}{1 + e^{-2x}}$')
\end{lstlisting}
\begin{figure}[H]
    \centering
        \includegraphics[scale=0.6]{\imgdir exp.eps}
\end{figure}
\item
$\displaystyle f(x) = \sqrt[3]{\frac{4-x^3}{1+x^2}}$
\vspace{0.2cm}

注意事項 : 
\begin{itemize}
\item
利用 \verb|np.cbrt()| 函式對函數開三分之一次方
\item
局部極大值(local maximum)\footnote{如果存在一個 $\epsilon > 0$,使得所有滿足 $|x - x^*| < \epsilon$ 的 $x$ 都有 $f(x^*)\geq f(x)$,我們就把點 $x^*$ 對應的函數值 $f(x^*)$ 稱為一個函數 $f$ 的局部極大值。} 為 1.587410 
\end{itemize}
\bigskip
\begin{lstlisting}[language = Python]
x = np.linspace(-5, 5, 100)
f = lambda x : np.cbrt((4 - x**3)/(1 + x**2))
fig = plt.figure()
plt.plot(x, f(x), color = 'g')
plt.grid(True)
plt.xlabel('X'), plt.ylabel('Y')
plt.text(0, f(0), 'x', color = 'r', fontsize = 15, \
    horizontalalignment = 'center', \
    verticalalignment = 'center')
plt.ylim([-2, 2])
plt.title(r'A local maximum is {:.6f} \
		at x = 0'.format(f(0)))
\end{lstlisting}
\begin{figure}[H]
    \centering
        \includegraphics[scale = 0.75]{\imgdir cbrt.eps}
\end{figure}
\item
$\displaystyle f(x) = \frac{1}{x}$
\vspace{0.2cm}

注意事項 : 
\begin{itemize}
\item
當 $x = 0$ 時,$f(x)$ 為 $\infty$,因此畫圖時需移除 $x = 0$
\item
利用 \verb|np.setdiff1d(ar1, ar2)| 移除兩組數列之差值
\item
水平漸進線為 $y = 0$,垂直漸進線為 $x = 0$
\end{itemize}
\bigskip
\begin{lstlisting}[language = Python]
def f(x): return 1/x
xfn = np.setdiff1d(np.linspace(-10, 0, 100),[0])  
xfp = np.setdiff1d(np.linspace(0, 10, 100),[0])   
plt.plot(xfn, f(xfn), color = 'c', linewidth = 3.0)
plt.plot(xfp, f(xfp), color = 'c', linewidth = 3.0)
plt.hlines(y = 0, xmin = -10, xmax = 10, \
	color = 'r', linestyles = '--', linewidth = 3.0)
plt.vlines(x = 0, ymin = -10, ymax = 10, \
	color = 'r', linestyles = '--', linewidth = 3.0)
ax = plt.gca() # 設定坐標軸之線
ax.spines['left'].set_position(('data', 0)) 
ax.spines['bottom'].set_position(('data', 0)) 
ax.spines['top'].set_color('none') 
ax.spines['right'].set_color('none') 
\end{lstlisting}
\begin{figure}[H]
    \centering
        \includegraphics[scale = 0.73]{\imgdir reciprocal.eps}
\end{figure}
\item
$\displaystyle f(x) = \frac{1}{2\sqrt{2\pi}}e^{\frac{-(x-1)^2}{8}}$
\vspace{0.2cm}

注意事項 : 
\begin{itemize}
\item
此函數為常態分配, $\mathcal{N}(\mu = 1, \,\,\sigma^2 = 4)$
\item
對稱於 $x = 1$,且絕對極大值為 0.199471
\item
水平漸進線為 $y = 0$
\end{itemize}

\begin{figure}[H]
    \centering
        \includegraphics[scale = 0.7]{\imgdir normal_123.eps}
\end{figure}
\item
$\displaystyle f(x) = \sqrt[3]{x^2}$
\vspace{0.2cm}

注意事項 : 
\begin{itemize}
\item
在設定 $x$ 時,提高生成的樣本數,畫出圖形之極值點較為明顯
\item
對稱於 $x = 0$,且絕對極小值為 0
\end{itemize}
\bigskip
\begin{lstlisting}[language = Python]
x = np.linspace(-10, 10, 1000)
f = lambda x : np.cbrt(x**2)
fig = plt.figure()
plt.plot(x, f(x), color = 'k')
plt.vlines(x = 0, ymin = 0, ymax = 10, \
		color = 'c', linestyles = '--')
plt.text(0, f(0), 'x', color = 'r', fontsize = 10, \
    horizontalalignment ='center', \
    verticalalignment ='center')
plt.grid(True)
plt.ylim([-0.1, 8])
plt.title(r'$f(x)= \sqrt[3]{x^2}$')
\end{lstlisting}
\begin{figure}[H]
    \centering
        \includegraphics[scale = 0.8]{\imgdir cbrt_1.eps}
\end{figure}
\bigskip
\item
$f(x) = 2x^3 - x^4$

注意事項 : 
\begin{itemize}
\item
利用 \verb|scipy| 套件中的 \verb|optimize| 找極值
\item
絕對極大值為 1.6875
\item
$(0, 0)$ 為反曲點
\end{itemize}
\bigskip
\begin{lstlisting}[language = Python]
import scipy.optimize as opt

x = np.linspace(-6, 6, 1000)
g = lambda x : 2*x**3 - x**4
fig = plt.figure(figsize = [6, 4])
plt.plot(x, g(x), 'purple')
f = lambda x : -g(x)
res = opt.minimize_scalar(f, bounds = (0, 2))
plt.text(res.x, -res.fun, 'x', color = 'r', \
	horizontalalignment = 'center', \
	verticalalignment = 'center')
plt.grid(True)
plt.ylim([-20, 4])
plt.title(r'The function maximum is {:.4f} at \
	x = {:.4f}'.format(-res.fun, res.x))
\end{lstlisting}
\begin{figure}[H]
    \centering
        \includegraphics[scale = 0.8]{\imgdir max.eps}
\end{figure}
\item
$\displaystyle f(x) = \frac{\ln x}{x^3}$
\vspace{0.2cm}

注意事項 : 
\begin{itemize}
\item
當 $x = 0$ 時,$f(x)$ 為 $-\infty$,因此畫圖時需移除 $x = 0$
\item
水平漸進線為 $y = 0$ , 垂直漸進線為 $x =0$ ,絕對極大值為 0.1226
\end{itemize}
\bigskip
\begin{lstlisting}[language = Python]
x = np.setdiff1d(np.linspace(0, 7, 500),[0])  
g = lambda x : np.log(x)/(x**3)
plt.plot(x, g(x), color = 'c')
f = lambda x : -g(x)
res = opt.minimize_scalar(f, bounds = (1, 2)) 
plt.text(res.x, -res.fun, 'x', color = 'r', \
	horizontalalignment = 'center', \
	verticalalignment = 'center')
plt.ylim([-3, 1])
ax = plt.gca()
ax.spines['left'].set_position(('data',0)) 
ax.spines['bottom'].set_position(('data',0)) 
ax.spines['top'].set_color('none')
ax.spines['right'].set_color('none')
plt.title(r'The function maximum is {:.4f} at \
	x = {:.4f}'.format(-res.fun, res.x))
\end{lstlisting}
\begin{figure}[H]
    \centering
        \includegraphics[scale = 0.8]{\imgdir log_1.eps}
\end{figure}
\item
$f(x) = 3,\quad 1 \leq x \leq 5$

注意事項 : 
\begin{itemize}
\item
利用 \verb|hlines(y, xmin, xmax)| 繪製水平線
\item
利用 \verb|vlines(x, ymin, ymax)| 繪製垂直線
\item
\verb|x|, \verb|y|, \verb|xmin|, \verb|xmax|, \verb|ymin|, \verb|ymax| 指的都是資料座標值
\end{itemize}
\bigskip
\begin{lstlisting}[language = Python]
# 繪製水平線
plt.hlines(y = 3, xmin = 1, xmax = 5, \
	color = 'k', linewidth = 2.0)
plt.title(r'$f(x) = 3, 1 \leq x \leq 5$')
# 繪製垂直線
plt.vlines(x = 3, ymin = 1, ymax = 5, \
	color = 'k', linewidth = 2.0)
plt.title(r'$x = 3, 1 \leq y \leq 5$')
\end{lstlisting}
\begin{figure}[H]
\centering
\includegraphics[scale = 0.48]{\imgdir hline.eps}
\includegraphics[scale = 0.48]{\imgdir vline.eps}
\end{figure}
\item
$x^2 + y^2 = 1$. 

注意事項 :
\begin{itemize}
\item 
將座標轉為極座標,$x = rcos(\theta)$ and $y = rsin(\theta), \,\, 0 \leq \theta \leq 2\pi$	 
\item
利用 \verb|axis('equal')| 將座標軸設定為等寬模式
\item
利用 \verb|ax.spines[''].set_position(())|,調整四個方向之坐標軸
\end{itemize}
\bigskip
\begin{lstlisting}[language = Python]
theta = np.linspace(0, 2*np.pi, 200) # 弧度
r = 1 # 半徑
x = r*np.cos(theta)
y = r*np.sin(theta)
plt.plot(x, y, color = 'g')
plt.ylim([-2, 2])
plt.title(r'$\mathrm{Parametric \,\,Equation \
	\,\,Circle}: \,\, x^2+y^2 = 1$')
plt.axis('equal')
ax = plt.gca()
ax.spines['left'].set_position(('data',0)) 
ax.spines['bottom'].set_position(('data',0)) 
ax.spines['top'].set_color('none')
ax.spines['right'].set_color('none')
\end{lstlisting}
\begin{figure}[H]
    \centering
        \includegraphics[scale = 0.8]{\imgdir circle.eps}
\end{figure}
\item 
Draw a square of size $= 1$

注意事項 :
\begin{itemize}
\item 
利用 \verb|patches| 模組中的 \verb|Rectangle| 繪畫長方形
\item
利用 \verb|axis('equal')| 將座標軸設定為等寬模式
\item
利用 \verb|ax.spines[''].set_position(())|,調整四個方向之坐標軸
\end{itemize}
\bigskip
\begin{lstlisting}[language = Python]
fig, ax = plt.subplots()
rec1 = pat.Rectangle((-0.5, -0.5), 1, 1, \
	edgecolor = 'm', fill = False)
ax.add_patch(rec1)
plt.xlim([-2, 2])
plt.ylim([-2, 2])
plt.axis("equal")
plt.title('A square of side 1')
ax = plt.gca()
ax.spines['left'].set_position(('data',0)) 
ax.spines['bottom'].set_position(('data',0)) 
ax.spines['top'].set_color('none')
ax.spines['right'].set_color('none')
\end{lstlisting}
\begin{figure}[H]
    \centering
        \includegraphics[scale = 0.7]{\imgdir square.eps}
\end{figure}
\end{enumerate}

\section{結語}
此次作品透過  Python 的繪圖套件「Matplotlib」 呈現散佈圖與折線圖,並嘗試更改其參數(顏色、樣式、粗細)等,最後透過 \LaTeX 編排 Python 程式碼與繪圖結果完成此次作品。

%\end{document}